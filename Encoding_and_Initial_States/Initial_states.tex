%% LyX 2.3.6.1 created this file.  For more info, see http://www.lyx.org/.
%% Do not edit unless you really know what you are doing.
\documentclass[english]{article}
\usepackage[a4paper, left=1cm, right=1cm, top=1cm, bottom=1.5cm]{geometry}
\usepackage[T1]{fontenc}
\usepackage[latin9]{inputenc}
\usepackage{amsmath}
\usepackage{babel}
\usepackage{fancyhdr}
\usepackage{datetime}

\pagestyle{fancy}
\fancyhf{}
\fancyfoot[R]{\newcommand{\mydate}{\the\day\ \monthname[\the\month]\ \the\year} \mydate\quad Shaohui Yao}
\fancyfoot[C]{\thepage}
\renewcommand{\headrulewidth}{0pt}
\renewcommand{\footrulewidth}{0pt}

\begin{document}
The initial states used in the computation processes.

--------------------------------

Fermi-Hubbard model with two lattice sites:

The following uses $\vert\alpha\rangle,\vert\beta\rangle,\vert\gamma\rangle,\dots$
to represent the ground state, first excited state, second excited
state, and so on. The subscripts denote the eigenvalues of the particle
number operator and the total spin $z$-component operator for each
eigenstate. For example, $\vert\alpha_{2,0}\rangle$ represents the
ground state with eigenvalue $\alpha_{2,0}$. 

\begin{eqnarray*}
\vert\psi^{+}\rangle & = & \frac{1}{\sqrt{2}}\left(\vert01\rangle+\vert10\rangle\right)\\
\vert\psi^{-}\rangle & = & \frac{1}{\sqrt{2}}\left(\vert01\rangle-\vert10\rangle\right)
\end{eqnarray*}

$\vert\alpha_{2,0}\rangle$: $n=2,M=0$

\begin{eqnarray*}
\vert\psi_{0}\rangle & = & \vert\psi^{+}\rangle\otimes\vert\psi^{+}\rangle\\
 & = & \frac{1}{\sqrt{2}}\left(\vert01\rangle+\vert10\rangle\right)\otimes\frac{1}{\sqrt{2}}\left(\vert01\rangle+\vert10\rangle\right)\\
 & = & \frac{1}{2}\left(\vert0101\rangle+\vert0110\rangle+\vert1001\rangle+\vert1010\rangle\right)
\end{eqnarray*}

$\vert\beta_{1,-1/2}\rangle$: $n=1,M=-\frac{1}{2}$

\begin{eqnarray*}
\vert\psi_{1}\rangle & = & \vert00\rangle\otimes\vert\psi^{+}\rangle\\
 & = & \frac{1}{\sqrt{2}}\left(\vert0001\rangle+\vert0010\rangle\right)
\end{eqnarray*}

$\vert\beta_{1,1/2}\rangle$: $n=1,M=\frac{1}{2}$

\begin{align*}
\vert\psi_{2}\rangle & =\vert\psi^{+}\rangle\otimes\vert00\rangle\\
 & =\frac{1}{\sqrt{2}}\left(\vert0100\rangle+\vert1000\rangle\right)
\end{align*}

$\vert\gamma_{2,-1}\rangle$: $n=2,M=-1$

\[
\vert\psi_{3}\rangle=\vert0011\rangle
\]

$\vert\gamma_{0,0}\rangle$: $n=0,M=0$

\[
\vert\psi_{4}\rangle=\vert0000\rangle
\]

$\vert\gamma_{2,1}\rangle$: $n=2,M=1$

\[
\vert\psi_{5}\rangle=\vert1100\rangle
\]

$\vert\gamma_{2,0}\rangle$: $n=2,M=0$

\begin{eqnarray*}
\vert\psi_{6}\rangle & = & \vert\psi^{+}\rangle\otimes\vert\psi^{-}\rangle\\
 & = & \frac{1}{\sqrt{2}}\left(\vert01\rangle+\vert10\rangle\right)\otimes\frac{1}{\sqrt{2}}\left(\vert01\rangle-\vert10\rangle\right)\\
 & = & \frac{1}{2}\left(\vert0101\rangle-\vert0110\rangle+\vert1001\rangle-\vert1010\rangle\right)
\end{eqnarray*}

$\vert\delta_{3,-1/2}\rangle$: $n=3,M=-\frac{1}{2}$

\begin{eqnarray*}
\vert\psi_{7}\rangle & = & \vert\psi^{+}\rangle\otimes\vert11\rangle\\
 & = & \frac{1}{\sqrt{2}}\left(\vert0111\rangle+\vert1011\rangle\right)
\end{eqnarray*}

$\vert\delta_{1,1/2}\rangle$: $n=1,M=\frac{1}{2}$

\begin{eqnarray*}
\vert\psi_{8}\rangle & = & \vert\psi^{-}\rangle\otimes\vert00\rangle\\
 & = & \frac{1}{\sqrt{2}}\left(\vert0100\rangle-\vert1000\rangle\right)
\end{eqnarray*}

$\vert\delta_{1,-1/2}\rangle$: $n=1,M=-\frac{1}{2}$

\begin{eqnarray*}
\vert\psi_{9}\rangle & = & \vert00\rangle\otimes\vert\psi^{-}\rangle\\
 & = & \frac{1}{\sqrt{2}}\left(\vert0001\rangle-\vert0010\rangle\right)
\end{eqnarray*}

$\vert\delta_{3,1/2}\rangle$: $n=3,M=\frac{1}{2}$

\begin{eqnarray*}
\vert\psi_{10}\rangle & = & \vert11\rangle\otimes\vert\psi^{+}\rangle\\
 & = & \frac{1}{\sqrt{2}}\left(\vert1101\rangle+\vert1110\rangle\right)
\end{eqnarray*}

$\vert\varepsilon_{2,0}\rangle$: $n=2,M=0$

\begin{eqnarray*}
\vert\psi_{11}\rangle & = & \vert\psi^{-}\rangle\otimes\vert\psi^{+}\rangle\\
 & = & \frac{1}{\sqrt{2}}\left(\vert01\rangle-\vert10\rangle\right)\otimes\frac{1}{\sqrt{2}}\left(\vert01\rangle+\vert10\rangle\right)\\
 & = & \frac{1}{2}\left(\vert0101\rangle+\vert0110\rangle-\vert1001\rangle-\vert1010\rangle\right)
\end{eqnarray*}

$\vert\zeta_{3,-1/2}\rangle$: $n=3,M=-\frac{1}{2}$

\begin{eqnarray*}
\vert\psi_{12}\rangle & = & \vert\psi^{-}\rangle\otimes\vert11\rangle\\
 & = & \frac{1}{\sqrt{2}}\left(\vert0111\rangle-\vert1011\rangle\right)
\end{eqnarray*}

$\vert\zeta_{3,1/2}\rangle$: $n=3,M=\frac{1}{2}$

\begin{eqnarray*}
\vert\psi_{13}\rangle & = & \vert11\rangle\otimes\vert\psi^{-}\rangle\\
 & = & \frac{1}{\sqrt{2}}\left(\vert1101\rangle-\vert1110\rangle\right)
\end{eqnarray*}

$\vert\eta_{2,0}\rangle$: $n=2,M=0$

\begin{eqnarray*}
\vert\psi_{14}\rangle & = & \vert\psi^{-}\rangle\otimes\vert\psi^{-}\rangle\\
 & = & \frac{1}{\sqrt{2}}\left(\vert01\rangle-\vert10\rangle\right)\otimes\frac{1}{\sqrt{2}}\left(\vert01\rangle-\vert10\rangle\right)\\
 & = & \frac{1}{2}\left(\vert0101\rangle-\vert0110\rangle-\vert1001\rangle+\vert1010\rangle\right)
\end{eqnarray*}

$\vert\theta_{4,0}\rangle$: $n=4,M=0$

\[
\vert\psi_{15}\rangle=\vert1111\rangle
\]

--------------------------------

Fermi-Hubbard model with four lattice sites:

The following uses $\vert\iota\rangle,\vert\kappa\rangle,\vert\lambda\rangle$
to represent the ground state, first excited state, second excited
state, and so on. The subscripts denote the eigenvalues of the particle
number operator and the total spin $z$-component operator for each
eigenstate. For example, $\vert\iota_{2,0}\rangle$ represents the
ground state with eigenvalue $\iota_{2,0}$. In the first excited
state, the subscripts 1 and 2 are labels used because no quantum numbers
were found to distinguish between different degenerate states, and
they have no physical meaning. In the paper, only the ground state
$\vert\iota_{2,0}\rangle$ and the second excited state $\vert\lambda_{4,0}\rangle$
are used. 

$\vert\iota_{2,0}\rangle$: $n=2,M=0$

\[
\vert\phi_{0}\rangle=\frac{1}{2}\left(\vert00\rangle\otimes\vert\psi^{+}\rangle+\vert\psi^{+}\rangle\otimes\vert00\rangle\right)\otimes\left(\vert00\rangle\otimes\vert\psi^{+}\rangle+\vert\psi^{+}\rangle\otimes\vert00\rangle\right)
\]

Specifically, 

\begin{eqnarray*}
\vert\phi_{0}\rangle & = & \frac{1}{\sqrt{16}}\left(\vert00\rangle\otimes\sqrt{2}\vert\psi^{+}\rangle\otimes\vert00\rangle\otimes\sqrt{2}\vert\psi^{+}\rangle\right.\\
 &  & +\vert00\rangle\otimes\sqrt{2}\vert\psi^{+}\rangle\otimes\sqrt{2}\vert\psi^{+}\rangle\otimes\vert00\rangle\\
 &  & +\sqrt{2}\vert\psi^{+}\rangle\otimes\vert00\rangle\otimes\vert00\rangle\otimes\sqrt{2}\vert\psi^{+}\rangle\\
 &  & +\left.\sqrt{2}\vert\psi^{+}\rangle\otimes\vert00\rangle\otimes\sqrt{2}\vert\psi^{+}\rangle\otimes\vert00\rangle\right)\\
 & = & \frac{1}{\sqrt{16}}\left[\vert00\rangle\otimes\sqrt{2}\vert\psi^{+}\rangle\otimes\left(\vert00\rangle\otimes\sqrt{2}\vert\psi^{+}\rangle+\sqrt{2}\vert\psi^{+}\rangle\otimes\vert00\rangle\right)\right.\\
 &  & +\left.\sqrt{2}\vert\psi^{+}\rangle\otimes\vert00\rangle\otimes\left(\vert00\rangle\otimes\sqrt{2}\vert\psi^{+}\rangle+\sqrt{2}\vert\psi^{+}\rangle\otimes\vert00\rangle\right)\right]\\
 & = & \frac{1}{2}\left(\vert00\rangle\otimes\vert\psi^{+}\rangle+\vert\psi^{+}\rangle\otimes\vert00\rangle\right)\otimes\left(\vert00\rangle\otimes\vert\psi^{+}\rangle+\vert\psi^{+}\rangle\otimes\vert00\rangle\right)\\
 & = & \frac{1}{4}\left(\vert00010001\rangle+\vert00010010\rangle+\vert00010100\rangle+\vert00011000\rangle\right.\\
 &  & +\vert00100001\rangle+\vert00100010\rangle+\vert00100100\rangle+\vert00101000\rangle\\
 &  & +\vert01000001\rangle+\vert01000010\rangle+\vert01000100\rangle+\vert01001000\rangle\\
 &  & +\left.\vert10000001\rangle+\vert10000010\rangle+\vert10000100\rangle+\vert10001000\rangle\right)
\end{eqnarray*}

$\vert\kappa_{3,1/2,1}\rangle$: $n=3,M=\frac{1}{2}$

\[
\vert\phi_{1}\rangle=\frac{1}{\sqrt{12}}\left(\vert00\rangle\otimes\vert11\rangle+\vert11\rangle\otimes\vert00\rangle+2\vert\psi^{+}\rangle\otimes\vert\psi^{+}\rangle\right)\otimes\left(\vert00\rangle\otimes\vert\psi^{+}\rangle+\vert\psi^{+}\rangle\otimes\vert00\rangle\right)
\]

Specifically, 

\begin{eqnarray*}
\vert\phi_{1}\rangle & = & \frac{1}{\sqrt{24}}\left(\vert00\rangle\otimes\vert11\rangle\otimes\vert00\rangle\otimes\sqrt{2}\vert\psi^{+}\rangle\right.\\
 &  & +\vert00\rangle\otimes\vert11\rangle\otimes\sqrt{2}\vert\psi^{+}\rangle\otimes\vert00\rangle\\
 &  & +\vert11\rangle\otimes\vert00\rangle\otimes\vert00\rangle\otimes\sqrt{2}\vert\psi^{+}\rangle\\
 &  & +\vert11\rangle\otimes\vert00\rangle\otimes\sqrt{2}\vert\psi^{+}\rangle\otimes\vert00\rangle\\
 &  & +\sqrt{2}\vert\psi^{+}\rangle\otimes\sqrt{2}\vert\psi^{+}\rangle\otimes\vert00\rangle\otimes\sqrt{2}\vert\psi^{+}\rangle\\
 &  & +\left.\sqrt{2}\vert\psi^{+}\rangle\otimes\sqrt{2}\vert\psi^{+}\rangle\otimes\sqrt{2}\vert\psi^{+}\rangle\otimes\vert00\rangle\right)\\
 & = & \frac{1}{\sqrt{24}}\left[\vert00\rangle\otimes\vert11\rangle\otimes\left(\vert00\rangle\otimes\sqrt{2}\vert\psi^{+}\rangle+\sqrt{2}\vert\psi^{+}\rangle\otimes\vert00\rangle\right)\right.\\
 &  & +\vert11\rangle\otimes\vert00\rangle\otimes\left(\vert00\rangle\otimes\sqrt{2}\vert\psi^{+}\rangle+\sqrt{2}\vert\psi^{+}\rangle\otimes\vert00\rangle\right)\\
 &  & +\left.\sqrt{2}\vert\psi^{+}\rangle\otimes\sqrt{2}\vert\psi^{+}\rangle\otimes\left(\vert00\rangle\otimes\sqrt{2}\vert\psi^{+}\rangle+\sqrt{2}\vert\psi^{+}\rangle\otimes\vert00\rangle\right)\right]\\
 & = & \frac{1}{\sqrt{12}}\left(\vert00\rangle\otimes\vert11\rangle+\vert11\rangle\otimes\vert00\rangle+2\vert\psi^{+}\rangle\otimes\vert\psi^{+}\rangle\right)\otimes\left(\vert00\rangle\otimes\vert\psi^{+}\rangle+\vert\psi^{+}\rangle\otimes\vert00\rangle\right)\\
 & = & \frac{1}{\sqrt{24}}\left(\vert00110001\rangle+\vert00110010\rangle+\vert00110100\rangle+\vert00111000\rangle\right.\\
 &  & +\vert11000001\rangle+\vert11000010\rangle+\vert11000100\rangle+\vert11001000\rangle\\
 &  & +\vert01010001\rangle+\vert01010010\rangle+\vert01010100\rangle+\vert01011000\rangle\\
 &  & +\vert01100001\rangle+\vert01100010\rangle+\vert01100100\rangle+\vert01101000\rangle\\
 &  & +\vert10010001\rangle+\vert10010010\rangle+\vert10010100\rangle+\vert10011000\rangle\\
 &  & +\left.\vert10100001\rangle+\vert10100010\rangle+\vert10100100\rangle+\vert10101000\rangle\right)
\end{eqnarray*}

$\vert\kappa_{3,-1/2,1}\rangle$: $n=3,M=-\frac{1}{2}$

\[
\vert\phi_{2}\rangle=\frac{1}{\sqrt{12}}\left(\vert00\rangle\otimes\vert\psi^{+}\rangle+\vert\psi^{+}\rangle\otimes\vert00\rangle\right)\otimes\left(\vert00\rangle\otimes\vert11\rangle+\vert11\rangle\otimes\vert00\rangle+2\vert\psi^{+}\rangle\otimes\vert\psi^{+}\rangle\right)
\]

Specifically, 

\begin{eqnarray*}
\vert\phi_{2}\rangle & = & \frac{1}{\sqrt{24}}\left(\vert00\rangle\otimes\sqrt{2}\vert\psi^{+}\rangle\otimes\vert00\rangle\otimes\vert11\rangle\right.\\
 &  & +\sqrt{2}\vert\psi^{+}\rangle\otimes\vert00\rangle\otimes\vert00\rangle\otimes\vert11\rangle\\
 &  & +\vert00\rangle\otimes\sqrt{2}\vert\psi^{+}\rangle\otimes\vert11\rangle\otimes\vert00\rangle\\
 &  & +\sqrt{2}\vert\psi^{+}\rangle\otimes\vert00\rangle\otimes\vert11\rangle\otimes\vert00\rangle\\
 &  & +\vert00\rangle\otimes\sqrt{2}\vert\psi^{+}\rangle\otimes\sqrt{2}\vert\psi^{+}\rangle\otimes\sqrt{2}\vert\psi^{+}\rangle\\
 &  & +\left.\sqrt{2}\vert\psi^{+}\rangle\otimes\vert00\rangle\otimes\sqrt{2}\vert\psi^{+}\rangle\otimes\sqrt{2}\vert\psi^{+}\rangle\right)\\
 & = & \frac{1}{\sqrt{24}}\left[\left(\vert00\rangle\otimes\sqrt{2}\vert\psi^{+}\rangle+\sqrt{2}\vert\psi^{+}\rangle\otimes\vert00\rangle\right)\otimes\vert00\rangle\otimes\vert11\rangle\right.\\
 &  & +\left(\vert00\rangle\otimes\sqrt{2}\vert\psi^{+}\rangle+\sqrt{2}\vert\psi^{+}\rangle\otimes\vert00\rangle\right)\otimes\vert11\rangle\otimes\vert00\rangle\\
 &  & +\left.\left(\vert00\rangle\otimes\sqrt{2}\vert\psi^{+}\rangle+\sqrt{2}\vert\psi^{+}\rangle\otimes\vert00\rangle\right)\otimes\sqrt{2}\vert\psi^{+}\rangle\otimes\sqrt{2}\vert\psi^{+}\rangle\right]\\
 & = & \frac{1}{\sqrt{12}}\left(\vert00\rangle\otimes\vert\psi^{+}\rangle+\vert\psi^{+}\rangle\otimes\vert00\rangle\right)\otimes\left(\vert00\rangle\otimes\vert11\rangle+\vert11\rangle\otimes\vert00\rangle+2\vert\psi^{+}\rangle\otimes\vert\psi^{+}\rangle\right)\\
 & = & \frac{1}{\sqrt{24}}\left(\vert00010011\rangle+\vert00100011\rangle+\vert01000011\rangle+\vert10000011\rangle\right.\\
 &  & +\vert00011100\rangle+\vert00101100\rangle+\vert01001100\rangle+\vert10001100\rangle\\
 &  & +\vert00010101\rangle+\vert00100101\rangle+\vert01000101\rangle+\vert10000101\rangle\\
 &  & +\vert00010110\rangle+\vert00100110\rangle+\vert01000110\rangle+\vert10000110\rangle\\
 &  & +\vert00011001\rangle+\vert00101001\rangle+\vert01001001\rangle+\vert10001001\rangle\\
 &  & +\left.\vert00011010\rangle+\vert00101010\rangle+\vert01001010\rangle+\vert10001010\rangle\right)
\end{eqnarray*}

$\vert\kappa_{3,-1/2,2}\rangle$: $n=3,M=-\frac{1}{2}$

\[
\vert\phi_{3}\rangle=\frac{1}{\sqrt{12}}\left(\vert00\rangle\otimes\vert\psi^{-}\rangle+\vert\psi^{-}\rangle\otimes\vert00\rangle\right)\otimes\left(\vert00\rangle\otimes\vert11\rangle+\vert11\rangle\otimes\vert00\rangle+2\vert\psi^{-}\rangle\otimes\vert\psi^{-}\rangle\right)
\]

Specifically, 

\begin{eqnarray*}
\vert\phi_{3}\rangle & = & \frac{1}{\sqrt{24}}\left(\vert00\rangle\otimes\sqrt{2}\vert\psi^{-}\rangle\otimes\vert00\rangle\otimes\vert11\rangle\right.\\
 &  & +\sqrt{2}\vert\psi^{-}\rangle\otimes\vert00\rangle\otimes\vert00\rangle\otimes\vert11\rangle\\
 &  & +\vert00\rangle\otimes\sqrt{2}\vert\psi^{-}\rangle\otimes\vert11\rangle\otimes\vert00\rangle\\
 &  & +\sqrt{2}\vert\psi^{-}\rangle\otimes\vert00\rangle\otimes\vert11\rangle\otimes\vert00\rangle\\
 &  & +\vert00\rangle\otimes\sqrt{2}\vert\psi^{-}\rangle\otimes\sqrt{2}\vert\psi^{-}\rangle\otimes\sqrt{2}\vert\psi^{-}\rangle\\
 &  & +\left.\sqrt{2}\vert\psi^{-}\rangle\otimes\vert00\rangle\otimes\sqrt{2}\vert\psi^{-}\rangle\otimes\sqrt{2}\vert\psi^{-}\rangle\right)\\
 & = & \frac{1}{\sqrt{12}}\left(\vert00\rangle\otimes\vert\psi^{-}\rangle+\vert\psi^{-}\rangle\otimes\vert00\rangle\right)\otimes\left(\vert00\rangle\otimes\vert11\rangle+\vert11\rangle\otimes\vert00\rangle+2\vert\psi^{-}\rangle\otimes\vert\psi^{-}\rangle\right)\\
 & = & \frac{1}{\sqrt{24}}\left(\vert00010011\rangle-\vert00100011\rangle+\vert01000011\rangle-\vert10000011\rangle\right.\\
 &  & +\vert00011100\rangle-\vert00101100\rangle+\vert01001100\rangle-\vert10001100\rangle\\
 &  & +\vert00010101\rangle-\vert00100101\rangle+\vert01000101\rangle-\vert10000101\rangle\\
 &  & -\vert00010110\rangle+\vert00100110\rangle-\vert01000110\rangle+\vert10000110\rangle\\
 &  & -\vert00011001\rangle+\vert00101001\rangle-\vert01001001\rangle+\vert10001001\rangle\\
 &  & +\left.\vert00011010\rangle-\vert00101010\rangle+\vert01001010\rangle-\vert10001010\rangle\right)
\end{eqnarray*}

$\vert\kappa_{3,1/2,2}\rangle$: $n=3,M=\frac{1}{2}$

\[
\vert\phi_{4}\rangle=\frac{1}{\sqrt{12}}\left(\vert00\rangle\otimes\vert11\rangle+\vert11\rangle\otimes\vert00\rangle+2\vert\psi^{-}\rangle\otimes\vert\psi^{-}\rangle\right)\otimes\left(\vert00\rangle\otimes\vert\psi^{-}\rangle+\vert\psi^{-}\rangle\otimes\vert00\rangle\right)
\]

\begin{eqnarray*}
\vert\phi_{4}\rangle & = & \frac{1}{\sqrt{24}}\left(\vert00\rangle\otimes\vert11\rangle\otimes\vert00\rangle\otimes\sqrt{2}\vert\psi^{-}\rangle\right.\\
 &  & +\vert00\rangle\otimes\vert11\rangle\otimes\sqrt{2}\vert\psi^{-}\rangle\otimes\vert00\rangle\\
 &  & +\vert11\rangle\otimes\vert00\rangle\otimes\vert00\rangle\otimes\sqrt{2}\vert\psi^{-}\rangle\\
 &  & +\vert11\rangle\otimes\vert00\rangle\otimes\sqrt{2}\vert\psi^{-}\rangle\otimes\vert00\rangle\\
 &  & +\sqrt{2}\vert\psi^{-}\rangle\otimes\sqrt{2}\vert\psi^{-}\rangle\otimes\vert00\rangle\otimes\sqrt{2}\vert\psi^{-}\rangle\\
 &  & +\left.\sqrt{2}\vert\psi^{-}\rangle\otimes\sqrt{2}\vert\psi^{-}\rangle\otimes\sqrt{2}\vert\psi^{-}\rangle\otimes\vert00\rangle\right)\\
 & = & \frac{1}{\sqrt{12}}\left(\vert00\rangle\otimes\vert11\rangle+\vert11\rangle\otimes\vert00\rangle+2\vert\psi^{-}\rangle\otimes\vert\psi^{-}\rangle\right)\otimes\left(\vert00\rangle\otimes\vert\psi^{-}\rangle+\vert\psi^{-}\rangle\otimes\vert00\rangle\right)\\
 & = & \frac{1}{\sqrt{24}}\left(\vert00110001\rangle-\vert00110010\rangle+\vert00110100\rangle-\vert00111000\rangle\right.\\
 &  & +\vert11000001\rangle-\vert11000010\rangle+\vert11000100\rangle-\vert11001000\rangle\\
 &  & +\vert01010001\rangle-\vert01010010\rangle+\vert01010100\rangle-\vert01011000\rangle\\
 &  & -\vert01100001\rangle+\vert01100010\rangle-\vert01100100\rangle+\vert01101000\rangle\\
 &  & -\vert10010001\rangle+\vert10010010\rangle-\vert10010100\rangle+\vert10011000\rangle\\
 &  & +\left.\vert10100001\rangle-\vert10100010\rangle+\vert10100100\rangle-\vert10101000\rangle\right)
\end{eqnarray*}

$\vert\lambda_{4,0}\rangle$: $n=4,M=0$

\[
\vert\phi_{5}\rangle=\frac{1}{6}\left(\vert00\rangle\otimes\vert11\rangle+\vert11\rangle\otimes\vert00\rangle+2\vert\psi^{+}\rangle\otimes\vert\psi^{+}\rangle\right)\otimes\left(\vert00\rangle\otimes\vert11\rangle+\vert11\rangle\otimes\vert00\rangle+2\vert\psi^{+}\rangle\otimes\vert\psi^{+}\rangle\right)
\]

\begin{eqnarray*}
\vert\phi_{5}\rangle & = & \frac{1}{\sqrt{36}}\left(\vert00\rangle\otimes\vert11\rangle\otimes\vert00\rangle\otimes\vert11\rangle\right.\\
 &  & +\vert00\rangle\otimes\vert11\rangle\otimes\vert11\rangle\otimes\vert00\rangle\\
 &  & +\vert00\rangle\otimes\vert11\rangle\otimes\sqrt{2}\vert\psi^{+}\rangle\otimes\sqrt{2}\vert\psi^{+}\rangle\\
 &  & +\vert11\rangle\otimes\vert00\rangle\otimes\vert00\rangle\otimes\vert11\rangle\\
 &  & +\vert11\rangle\otimes\vert00\rangle\otimes\vert11\rangle\otimes\vert00\rangle\\
 &  & +\vert11\rangle\otimes\vert00\rangle\otimes\sqrt{2}\vert\psi^{+}\rangle\otimes\sqrt{2}\vert\psi^{+}\rangle\\
 &  & +\sqrt{2}\vert\psi^{+}\rangle\otimes\sqrt{2}\vert\psi^{+}\rangle\otimes\vert00\rangle\otimes\vert11\rangle\\
 &  & +\sqrt{2}\vert\psi^{+}\rangle\otimes\sqrt{2}\vert\psi^{+}\rangle\otimes\vert11\rangle\otimes\vert00\rangle\\
 &  & +\left.\sqrt{2}\vert\psi^{+}\rangle\otimes\sqrt{2}\vert\psi^{+}\rangle\otimes\sqrt{2}\vert\psi^{+}\rangle\otimes\sqrt{2}\vert\psi^{+}\rangle\right)\\
 & = & \frac{1}{6}\left(\vert00\rangle\otimes\vert11\rangle\otimes\left(\vert00\rangle\otimes\vert11\rangle+\vert11\rangle\otimes\vert00\rangle+2\vert\psi^{+}\rangle\otimes\vert\psi^{+}\rangle\right)\right.\\
 &  & +\vert11\rangle\otimes\vert00\rangle\otimes\left(\vert00\rangle\otimes\vert11\rangle+\vert11\rangle\otimes\vert00\rangle+2\vert\psi^{+}\rangle\otimes\vert\psi^{+}\rangle\right)\\
 &  & +\left.2\vert\psi^{+}\rangle\otimes\vert\psi^{+}\rangle\otimes\left(\vert00\rangle\otimes\vert11\rangle+\vert11\rangle\otimes\vert00\rangle+2\vert\psi^{+}\rangle\otimes\vert\psi^{+}\rangle\right)\right)\\
 & = & \frac{1}{6}\left(\vert00\rangle\otimes\vert11\rangle+\vert11\rangle\otimes\vert00\rangle+2\vert\psi^{+}\rangle\otimes\vert\psi^{+}\rangle\right)\otimes\left(\vert00\rangle\otimes\vert11\rangle+\vert11\rangle\otimes\vert00\rangle+2\vert\psi^{+}\rangle\otimes\vert\psi^{+}\rangle\right)\\
 & = & \frac{1}{6}\left(\vert00110011\rangle+\vert00111100\rangle+\vert00110101\rangle+\vert00110110\rangle+\vert00111001\rangle+\vert00111010\rangle\right.\\
 &  & +\vert11000011\rangle+\vert11001100\rangle+\vert11000101\rangle+\vert11000110\rangle+\vert11001001\rangle+\vert11001010\rangle\\
 &  & +\vert01010011\rangle+\vert01011100\rangle+\vert01010101\rangle+\vert01010110\rangle+\vert01011001\rangle+\vert01011010\rangle\\
 &  & +\vert01100011\rangle+\vert01101100\rangle+\vert01100101\rangle+\vert01100110\rangle+\vert01101001\rangle+\vert01101010\rangle\\
 &  & +\vert10010011\rangle+\vert10011100\rangle+\vert10010101\rangle+\vert10010110\rangle+\vert10011001\rangle+\vert10011010\rangle\\
 &  & +\left.\vert10100011\rangle+\vert10101100\rangle+\vert10100101\rangle+\vert10100110\rangle+\vert10101001\rangle+\vert10101010\rangle\right)
\end{eqnarray*}

\end{document}
